\documentclass[11pt]{article}
\usepackage[utf8]{inputenc}
\usepackage{fullpage}
\usepackage{authblk}
\usepackage{url}
\usepackage{physics}
\usepackage{amsmath}
\usepackage{amssymb}
\usepackage{float}
\usepackage{tikz}
\usepackage{hyperref}
\usepackage{verbatim}
\usepackage[font=small,labelfont=bf, skip=1pt,center]{caption}
\DeclareMathOperator*{\minimize}{minimize}
\usepackage{bm}
\usepackage{booktabs}
\usepackage{tabularx}

\newcommand{\normnew}[2]{\left \lVert #1 \right \rVert_{#2}}

\counterwithout{figure}{section}


\title{Image-to-world transforms in \texttt{ips\_cam}}

\author{Stuart Johnson}
\affil{stuart.g.johnson@gmail.com}

\date{\today}

\begin{document}

\maketitle

\texttt{ips\_cam} assumes that it is viewing multiple ARUCO tags, each at a given height above the floor, and parallel to the plane of the floor. Furthermore, it assumes it is given the camera intrinsic ($K$) and extrinsic matrix ($E$). The intrinsic matrix is the result of camera monocular calibration, and the extrinsic matrix is obtained from an image of a chessboard pattern which defines the world coordinate system (AKA the Indoor Coordinate System or ICS). A step in finding ARUCO tags in the ICS is a conversion of image points to world coordinates.

Writing the full camera projection matrix:

\begin{equation}
P = K E
\end{equation}

we have the pinhole camera model:

\begin{equation}
P \begin{bmatrix}
X \\
Y \\
Z \\
1
\end{bmatrix}
= \lambda \begin{bmatrix}
u \\
v \\
1
\end{bmatrix}
\end{equation}

where $[X,Y,Z]$ are world (ICS) coordinates and $[u,v]$ are image (pixel) coordinates. Note that $P \in \mathbb{R}^{3 \times 4}$ and $\lambda$ is a scale factor to be determined by perspective divide. We can extract the third column of $P$ as $P_z$, leaving the remainder of the matrix as $P_{xy}$. We now have:

\begin{equation}
P_{xy} \begin{bmatrix}
X \\
Y \\
1
\end{bmatrix}
+
P_z Z
= \lambda \begin{bmatrix}
u \\
v \\
1
\end{bmatrix}
\end{equation}

$P_{xy} \in \mathbb{R}^{3 \times 3}$ is quite invertible, so we can write, after multiplying on the left by $P^{-1}_{xy}$:

\begin{equation}
\label{eqn:splitz}
\begin{bmatrix}
X \\
Y \\
1
\end{bmatrix}
+
P^{-1}_{xy} P_z
= \lambda P^{-1}_{xy} \begin{bmatrix}
u \\
v \\
1
\end{bmatrix}
\end{equation}

We now note that $\lambda$ is entirely determined by the third row of this equation:

\begin{equation}
1
+
\left(P^{-1}_{xy} P_z \right)_{3,:}
= \lambda \left( P^{-1}_{xy} \begin{bmatrix}
u \\
v \\
1
\end{bmatrix}
\right)_{3,:}
\end{equation}

where the notation $(M)_{3,:}$ refers to the third row of matrix M. Note that while this example yields a scalar equation in $\lambda$, we typically are solving for 4 corners of an ARUCO tag at once, and so we have an expression for $\lambda$ for each point (a row vector equation, if you will). After solving for each $\lambda$ we then solve for world coordinates using a version of equation \ref{eqn:splitz}:

\begin{equation}
\begin{bmatrix}
X \\
Y \\
1
\end{bmatrix}
=
\lambda P^{-1}_{xy} \begin{bmatrix}
u \\
v \\
1
\end{bmatrix}
- P^{-1}_{xy} P_z
\end{equation}

Which is to say, we are correcting for an error caused by the perspective shift of ARUCO tags at non-zero $Z$ (height above the floor).

All calculations in the code are carried out with \texttt{OPENCV}'s \texttt{cv::Mat} class - with some custom extensions for matrix row and column surgery.


\end{document}